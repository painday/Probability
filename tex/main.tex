% 模板出处:https://zhuanlan.zhihu.com/p/379360037

\documentclass[12pt, a4paper, oneside]{book}
\usepackage{ctex}
\usepackage{amsmath, amsthm, amssymb, bm, graphicx, hyperref, mathrsfs}
\hypersetup{
    colorlinks=true,
    linkcolor=blue,
    filecolor=blue,      
    urlcolor=blue,
    citecolor=cyan,
}
\usepackage{comment}
\usepackage{titlesec} % 添加 titlesec 宏包
\titleclass{\section}{straight} % 清除默认章节格式

% 章节格式统一设置
\titleformat{\section}[runin]
  {\normalfont\bfseries}
  {\thesection.}
  {-0.5em}
  {\quad}
  
\titleformat{\subsection}[runin]
  {\normalfont\bfseries}
  {\thesubsection.}
  {-0.3em}
  {\quad}

% 间距微调
\titlespacing*{\section}
  {0pt}
  {1.2ex plus 0.5ex minus 0.2ex}
  {0.3em}

\titlespacing*{\subsection}
  {15pt} % 二级标题增加缩进
  {0.8ex plus 0.3ex minus 0.1ex}
  {0.2em}

\renewcommand{\thesection}{\arabic{section}}
\renewcommand{\thesubsection}{\arabic{section}\arabic{subsection}}



% 定理环境样式调整
\theoremstyle{definition}
\newtheorem{theorem}{定理}[section]
\newtheorem{lemma}[theorem]{引理}
\newtheorem{definition}[theorem]{定义}
\newtheorem{example}[theorem]{例}
\newtheorem{proposition}[theorem]{命题}


\title{{\Huge{\textbf{测度与概率 Notes}}}}
\author{Liu Fuzhou}
\date{\today}
\linespread{1.5}






\begin{document}

\maketitle

\pagenumbering{roman}
\setcounter{page}{1}

%\begin{comment}
\begin{center}
    \Huge\textbf{前言}
\end{center}~\

要看偏微分方程,就要先看泛函分析. 要看泛函分析,就要先看实分析. 泛函分析和实分析对随机分析也是重要的. 无论如何也绕不开 $L^p$ 空间. 可见掌握实分析十分重要. 以前其实也看过这个主题的书,结果后来都忘了. 看来做一些有形的笔记是很有必要的. 不需要特别详细,提纲挈领即可.

主要参考文献:
\begin{enumerate}
\item[1.] 近代概率引论:测度、鞅和随机微分方程. 袁震东. 科学出版社, 1991.
\item[2.] 测度与概率教程. 任佳刚,巫静. 科学出版社, 2018.
\item[3.] Measure, Integration \& Real Analysis. Sheldon Axler. Springer, 2020.
\end{enumerate}

~\\
\begin{flushright}
    \begin{tabular}{c}
        Liu Fuzhou\\
        \today
    \end{tabular}
\end{flushright}
%\end{comment}

\newpage
\pagenumbering{Roman}
\setcounter{page}{1}
\tableofcontents
\newpage
\setcounter{page}{1}
\pagenumbering{arabic}


\section{集列的极限}
首先比较重要的概念就是集列的上极限与下极限. 集列 $\{A_n\}_{n\in\mathbb N}$ 的上极限就是落到其中无限多个集合的元素所做成的集合,也即 $\limsup_{n\to\infty}A_n=\bigcap_{n=1}^\infty\bigcup_{k=n}^\infty A_k.$ 
下极限就是除去有限多个集合后,落到所有集中的元素所做成的集合,也即 $\liminf_{n\to\infty}A_n=\bigcup_{n=1}^\infty\bigcap_{k=n}^\infty A_k.$ 换言之也即“最终落到集列” $A_n$ 中的那些元素之全体.
显然,下限集中的元素也都落到上限集之中,也即 $\liminf A_n\subset \limsup A_n.$ 如果集列 $A_n$ 的上限集与下限集相等,那么就说集列 $A_n$ 收敛,并称 $A=\liminf A_n=\limsup A_n$ 为集列 $A_n$ 的极限. 这个极限是关于集合包含的偏序关系说的. 
与实数的情况类似,有单调集列的收敛定理:如果 $A_n\uparrow$ 是单调递增的,那么 $\lim A_n=\bigcup_n A_n.$ 类似地,如果 $A_n\downarrow$ 是单调递减的,那么 $\lim A_n=\bigcap_n A_n.$ 


\begin{example}
    设 $\{A_n\}_{n\in\mathbb N}$ 为单调减少集列. 证明 
    \begin{equation*}
        A_1=\sum_{n=1}^\infty (A_n-A_{n+1}) + \bigcap_{n=1}^\infty A_n
    \end{equation*}
\end{example}
\begin{proof}
首先,右边的每个集合都含于 $A_1.$ 因此右边含于 $A_1.$ 其次,若 $x\in A_1$ 且 $x\notin \bigcap_{n=1}^\infty A_n$, 则 $x\in \sum_{n=1}^\infty (A_n-A_{n-1})$ 且 $x\notin \bigcap_{n=1}^\infty A_n.$ 因此 $A_1$ 也是单调递减的. 最后,若 $x\in \bigcap_{n=1}^\infty A_n$, 则存在某个 $n>1,$ 使得 $x\notin A_n.$ 令 $n_0$ 为最小的这种 $n,$ 那么就有 $x\notin A_{n_0},\ x\in A_{n_0-1}.$ 于是就有 $x\in \sum_{n=1}^\infty (A_n-A_{n-1}).$ 这就证明了 
$A_1\subset \sum_{n=1}^\infty (A_n-A_{n+1}) + \bigcap_{n=1}^\infty A_n.$ 
\end{proof}

\section{Riemann 积分} 
还是用 \cite{Axler_2020} 作为参考书比较好,还是彩色的. 我还是比较喜欢 Riemann 积分的,最开始是在梅加强的书上学的. 比较重要的技术就是证明 Darboux 上和总是大于等于 Darboux 下和那里,需要用到两个 partition 的 merge. 也就是说,用到了有界闭区间的 partition 全体构成一个 directed set 的性质. Darboux 上和与 Darboux 下和之差就是所谓的函数振幅 
$\Omega_\mathcal P (f)=\sum_{i} (\sup_{x_{i-1}\leq x\leq x_i} f-\inf_{x_{i-1}\leq x\leq x_i})(x_i-x_{i-1}),$ 这里 $\mathcal P=\{a=x_0<x_1<\cdots<x_n=b\}$ 是所讨论的具体的分割. 

如果向 partition 的点集增加元素,也即使得分割变得更细,那么结果就是上和不增,下和不减. 这就导致了两个极限 $\lim_{\mathcal P} \mathcal L(f,\mathcal P,[a,b])$ 和 $\lim_{\mathcal P} \mathcal U(f,\mathcal P,[a,b]),$ 称为 Riemann 下积分 $\underline{\int}$ 和上积分 $\overline{\int}$. 这两个极限是关于 partition 全体所具有的 directed set 结构说的, 也就是说是 net 的极限. 由于下和不大于上和,也即 $\mathcal L\leq \mathcal U,$ 因此这两个极限之间也有关系
\begin{equation}
    \underline{\int}_a^b f(x)\ \mathrm dx\leq  \overline{\int}_a^b f(x)\ \mathrm dx
\end{equation}
如果 Riemann 上下积分相等,就说 $f$ 在 $[a,b]$ 上 Riemann 可积,其积分 $\int_a^b f(x)\ \mathrm dx$ 就定义为上下积分的共同值. 

任意分割 $\mathcal P$ 都将区间 $[a,b]$ 分成 $\#(\mathcal P)-1$ 个首尾相接的闭区间 $[x_0,x_1],\cdots,[x_{n-1},x_n].$ 从每个闭区间挑选一个元素出来,就可以做成一个集合 $\Delta.$ 这种挑选当然是不唯一的. 称这种通过在每个闭区间中挑出一个元素所形成的集合为与 partition $\mathcal P$ 相伴的一个点集. 
设 $\{\mathcal P_n\}_{n\in\mathbb N}$ 是任意一列 partition, 并且 $\{\Delta_n\}_{n\in\mathbb N}$ 是一列与之相伴的点集,每个 $\Delta_n$ 中元素形如 $\Delta_n=\{\xi_1,\cdots,\xi_{\#(\mathcal P_n)-1}\},$ 那么就可以考虑 Riemann 和所形成的数列
\begin{equation}
    \mathcal R_n:=\mathcal R(f,\mathcal P_n,\Delta_n,[a,b])=\sum_{i=1}^{\#(\mathcal P_n)-1} f(\xi_i)(x_{i}-x_{i-1})
\end{equation}
显然有 $\mathcal L(f,\mathcal P_n,[a,b])\leq \mathcal R_n\leq \mathcal U(f,\mathcal P_n,[a,b]).$ 因此有
\begin{equation}
    \underline{\int}_a^b f(x)\ \mathrm dx\leq \liminf \mathcal R_n\leq \limsup \mathcal R_n\leq  \overline{\int}_a^b f(x)\ \mathrm dx
\end{equation}

这就表明,如果 $\mathcal R_n$ 有收敛子列的话,那么该子列的极限就必定位于上下积分之间.  当然了,如果 $f$ 是 Riemann 可积的,这个极限就必定等于 $\int_a^b f(x)\ \mathrm dx.$ 


以上讨论了利用 partition 的全体所组成的集合之上的 directed set 结构来定义的 Riemann 积分. 下面考虑利用 Riemann 和关于 partition 的 mesh 的极限的途径. 这就是说,把 Riemann 积分定义为极限
\begin{equation}\label{eq:Riemann_limit}
    \lim_{\|\mathcal P\|\to 0} \mathcal R(f,\mathcal P,\Delta,[a,b])
\end{equation}
其中 $\|\mathcal P\|=\max_{i}|x_{i}-x_{i-1}|$ 是 partition 的 mesh. 这个极限其实相当强,因为其定义只涉及 partition, 对相伴的点集 $\Delta$ 没什么要求.  

要证明两种定义的等价性,首先要注意到,上下积分之差恰好等于函数振幅的下确界 $\inf_{\mathcal P}\Omega_\mathcal P(f).$ 这就是说 $\lim_{\mathcal P} (\mathcal U(f,\mathcal P,[a,b])-\mathcal L(f,\mathcal P,[a,b]))=\inf_{\mathcal P}\Omega_\mathcal P(f),$ 或者说 $\lim_{\mathcal P}\Omega_\mathcal P(f)=\inf_{\mathcal P}\Omega_\mathcal P(f).$ 
事实上,设 $\varepsilon>0,$ 那么由 net 的极限的定义,存在 partition $\mathcal P_0,$ 使得任意细于 $\mathcal P_0$ 的分割 $\mathcal P$ 都满足 $|\Omega_{\mathcal P_0}(f)-\lim_{\mathcal P} \Omega_{\mathcal P}(f)|<\varepsilon.$ 
另一方面,由下确界的定义,存在 partition $\mathcal P_1$ 使得 $\Omega_{\mathcal P_1}(f)< \inf_{\mathcal P}\Omega_{\mathcal P}(f)+\varepsilon.$
取分割 $\mathcal P_2$ 为 $\mathcal P_0$ 与 $\mathcal P_1$ 的 merge, 那么就有 
\begin{equation}
    \lim_{\mathcal P}\Omega_{\mathcal P}(f)-\varepsilon< \Omega_{\mathcal P_2}(f)\leq \Omega_{\mathcal P_1}(f)<\inf_{\mathcal P}\Omega_{\mathcal P}(f)+\varepsilon
\end{equation}
于是 $\inf_{\mathcal P}\Omega_{\mathcal P}(f)\leq \lim_{\mathcal P}\Omega_{\mathcal P}(f)<\inf_{\mathcal P}\Omega_{\mathcal P}(f)+2\varepsilon.$ 这就证明了 
\begin{equation}
    \overline{\int}_a^b f(x)\ \mathrm dx - \underline{\int}_a^b f(x)\ \mathrm dx = \inf_{\mathcal P}\Omega_{\mathcal P}(f)
\end{equation}

假如极限 \eqref{eq:Riemann_limit} 存在,那么

\nocite{*}
\bibliographystyle{plain}
\bibliography{../Ref/reference.bib}

\end{document}
