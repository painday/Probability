% 模板出处:https://zhuanlan.zhihu.com/p/379360037

\documentclass[12pt, a4paper, oneside]{ctexbook}
\usepackage{amsmath, amsthm, amssymb, bm, graphicx, hyperref, mathrsfs}
\usepackage{comment}
\usepackage{titlesec} % 添加 titlesec 宏包

\titleformat{\section} % 重新定义 section 格式
  {\normalfont\Large\bfseries} % 标题格式
  {\thesection} % 标签
  {1em} % 标签和标题之间的间距
  {} % 前代码

\title{{\Huge{\textbf{测度与概率 Notes}}}}
\author{Liu Fuzhou}
\date{\today}
\linespread{1.5}
\newtheorem{theorem}{定理}[section]
\newtheorem{definition}[theorem]{定义}
\newtheorem{lemma}[theorem]{引理}
\newtheorem{corollary}[theorem]{推论}
\newtheorem{example}[theorem]{例}
\newtheorem{proposition}[theorem]{命题}

\begin{document}

\maketitle

\pagenumbering{roman}
\setcounter{page}{1}

%\begin{comment}
\begin{center}
    \Huge\textbf{前言}
\end{center}~\

以前其实也看过这个主题的书,结果后来都忘了. 看来做一些有形的笔记是很有必要的. 不需要特别详细,提纲挈领即可.

主要参考文献:
\begin{enumerate}
\item[1.] 近代概率引论:测度、鞅和随机微分方程. 袁震东. 科学出版社, 1991.
\end{enumerate}
~\\
\begin{flushright}
    \begin{tabular}{c}
        Liu Fuzhou\\
        \today
    \end{tabular}
\end{flushright}
%\end{comment}

\newpage
\pagenumbering{Roman}
\setcounter{page}{1}
\tableofcontents
\newpage
\setcounter{page}{1}
\pagenumbering{arabic}

\chapter{测度论}
 
\section{$\sigma$-代数与测度}

首先比较重要的概念就是集列的上极限与下极限. 集列 $\{A_n\}_{n\in\mathbb N}$ 的上极限就是落到其中无限多个集合的元素所做成的集合,也即 $\limsup_{n\to\infty}A_n=\bigcap_{n=1}^\infty\bigcup_{k=n}^\infty A_k.$ 
下极限就是除去有限多个集合后,落到所有集中的元素所做成的集合,也即 $\liminf_{n\to\infty}A_n=\bigcup_{n=1}^\infty\bigcap_{k=n}^\infty A_k.$ 换言之也即“最终落到集列” $A_n$ 中的那些元素之全体.
显然,下限集中的元素也都落到上限集之中,也即 $\liminf A_n\subset \limsup A_n.$ 如果集列 $A_n$ 的上限集与下限集相等,那么就说集列 $A_n$ 收敛,并称 $A=\liminf A_n=\limsup A_n$ 为集列 $A_n$ 的极限.      这个极限是关于集合包含的偏序关系说的. 
与实数的情况类似,有单调集列的收敛定理:如果 $A_n\uparrow$ 是单调递增的,那么 $\lim A_n=\bigcup_n A_n.$ 类似地,如果 $A_n\downarrow$ 是单调递减的,那么 $\lim A_n=\bigcap_n A_n.$ 


\begin{example}
    设 $\{A_n\}_{n\in\mathbb N}$ 为单调减少集列. 证明 
    \begin{equation*}
        A_1=\sum_{n=1}^\infty (A_n-A_{n+1}) + \bigcap_{n=1}^\infty A_n
    \end{equation*}
\end{example}
\begin{proof}
首先,右边的每个集合都含于 $A_1.$ 因此右边含于 $A_1.$ 其次,若 $x\in A_1$ 且 $x\notin \bigcap_{n=1}^\infty A_n$, 则 $x\in \sum_{n=1}^\infty (A_n-A_{n-1})$ 且 $x\notin \bigcap_{n=1}^\infty A_n.$ 因此 $A_1$ 也是单调递减的. 最后,若 $x\in \bigcap_{n=1}^\infty A_n$, 则存在某个 $n>1,$ 使得 $x\notin A_n.$ 令 $n_0$ 为最小的这种 $n,$ 那么就有 $x\notin A_{n_0},\ x\in A_{n_0-1}.$ 于是就有 $x\in \sum_{n=1}^\infty (A_n-A_{n-1}).$ 这就证明了 
$A_1\subset \sum_{n=1}^\infty (A_n-A_{n+1}) + \bigcap_{n=1}^\infty A_n.$ 
\end{proof}



\end{document}